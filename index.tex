\documentclass[]{article}
\usepackage{lmodern}
\usepackage{amssymb,amsmath}
\usepackage{ifxetex,ifluatex}
\usepackage{fixltx2e} % provides \textsubscript
\ifnum 0\ifxetex 1\fi\ifluatex 1\fi=0 % if pdftex
  \usepackage[T1]{fontenc}
  \usepackage[utf8]{inputenc}
\else % if luatex or xelatex
  \ifxetex
    \usepackage{mathspec}
  \else
    \usepackage{fontspec}
  \fi
  \defaultfontfeatures{Ligatures=TeX,Scale=MatchLowercase}
\fi
% use upquote if available, for straight quotes in verbatim environments
\IfFileExists{upquote.sty}{\usepackage{upquote}}{}
% use microtype if available
\IfFileExists{microtype.sty}{%
\usepackage{microtype}
\UseMicrotypeSet[protrusion]{basicmath} % disable protrusion for tt fonts
}{}
\usepackage[margin=1in]{geometry}
\usepackage{hyperref}
\hypersetup{unicode=true,
            pdftitle={Weekly Progress Reports},
            pdfauthor={Travis Sondgerath},
            pdfborder={0 0 0},
            breaklinks=true}
\urlstyle{same}  % don't use monospace font for urls
\usepackage{graphicx,grffile}
\makeatletter
\def\maxwidth{\ifdim\Gin@nat@width>\linewidth\linewidth\else\Gin@nat@width\fi}
\def\maxheight{\ifdim\Gin@nat@height>\textheight\textheight\else\Gin@nat@height\fi}
\makeatother
% Scale images if necessary, so that they will not overflow the page
% margins by default, and it is still possible to overwrite the defaults
% using explicit options in \includegraphics[width, height, ...]{}
\setkeys{Gin}{width=\maxwidth,height=\maxheight,keepaspectratio}
\IfFileExists{parskip.sty}{%
\usepackage{parskip}
}{% else
\setlength{\parindent}{0pt}
\setlength{\parskip}{6pt plus 2pt minus 1pt}
}
\setlength{\emergencystretch}{3em}  % prevent overfull lines
\providecommand{\tightlist}{%
  \setlength{\itemsep}{0pt}\setlength{\parskip}{0pt}}
\setcounter{secnumdepth}{0}
% Redefines (sub)paragraphs to behave more like sections
\ifx\paragraph\undefined\else
\let\oldparagraph\paragraph
\renewcommand{\paragraph}[1]{\oldparagraph{#1}\mbox{}}
\fi
\ifx\subparagraph\undefined\else
\let\oldsubparagraph\subparagraph
\renewcommand{\subparagraph}[1]{\oldsubparagraph{#1}\mbox{}}
\fi

%%% Use protect on footnotes to avoid problems with footnotes in titles
\let\rmarkdownfootnote\footnote%
\def\footnote{\protect\rmarkdownfootnote}

%%% Change title format to be more compact
\usepackage{titling}

% Create subtitle command for use in maketitle
\providecommand{\subtitle}[1]{
  \posttitle{
    \begin{center}\large#1\end{center}
    }
}

\setlength{\droptitle}{-2em}

  \title{Weekly Progress Reports}
    \pretitle{\vspace{\droptitle}\centering\huge}
  \posttitle{\par}
    \author{Travis Sondgerath}
    \preauthor{\centering\large\emph}
  \postauthor{\par}
      \predate{\centering\large\emph}
  \postdate{\par}
    \date{2019-11-09}


\begin{document}
\maketitle

{
\setcounter{tocdepth}{2}
\tableofcontents
}
\hypertarget{april-1st---april-7th-2019}{%
\section{April 1st - April 7th, 2019}\label{april-1st---april-7th-2019}}

\hypertarget{progress-items}{%
\subsection{Progress Items}\label{progress-items}}

\begin{itemize}
\tightlist
\item
  Created an RStudio Cloud
  \textbf{\href{https://travis-shinin-spot.shinyapps.io/new_equip_form/}{lesson}}
  on maintaining and changing the code for the new equipment
  \textbf{\href{https://travis-shinin-spot.shinyapps.io/new_equip_form/}{form}}.
\item
  Plan to also create one for making equipment life books and generating
  a basic report using equipment data.
\end{itemize}

\hypertarget{agenda-items}{%
\subsection{Agenda Items}\label{agenda-items}}

\begin{enumerate}
\def\labelenumi{\arabic{enumi}.}
\tightlist
\item
  I would suggest instead of having our regular call at the normal time
  on Wednesdays, that we see if the larger group would like a
  demonstration on a different day. Perhaps either at 7 am or 11 am?
\end{enumerate}

\hypertarget{march-25th---march-31st-2019}{%
\section{March 25th, - March 31st,
2019}\label{march-25th---march-31st-2019}}

\hypertarget{progress-items-1}{%
\subsection{Progress Items}\label{progress-items-1}}

\begin{itemize}
\tightlist
\item
  Created mandatory fields in the
  \textbf{\href{https://travis-shinin-spot.shinyapps.io/new_equip_form/}{form}}
  for new equipment.
\item
  Finished annotations for RStudio Cloud
  \textbf{\href{https://rstudio.cloud/project/255745}{workspace}} lesson
  on maintaining the eTool code
\item
  Updated final report.
\end{itemize}

\hypertarget{agenda-items-1}{%
\subsection{Agenda Items}\label{agenda-items-1}}

\begin{enumerate}
\def\labelenumi{\arabic{enumi}.}
\tightlist
\item
  Wanted to make sure to let everyone know that I was compensated for
  January/February work. Have invoiced Mimi for March.
\item
  Discuss date proposals for larger group discussion. Good time to
  update them on progress.
\item
  Discuss RStudio Cloud Workspace.
\end{enumerate}

\hypertarget{march-18th---march-24th-2019}{%
\section{March 18th - March 24th,
2019}\label{march-18th---march-24th-2019}}

\hypertarget{progress-items-2}{%
\subsection{Progress Items}\label{progress-items-2}}

\begin{itemize}
\tightlist
\item
  Made revisions to the
  \textbf{\href{https://travis-shinin-spot.shinyapps.io/new_equip_form/}{form}}
  for new equipment. It is now possible for some of the drop downs to
  type in an additional option if not already shown.
\item
  Created a
  \textbf{\href{https://github.com/paceafenet/etool_dev/blob/master/New\%20Equipment\%20Form\%20User\%20Guide.pdf}{user
  guide}} for the form referenced above.
\item
  Added more demo courses to the project RStudio Cloud
  \textbf{\href{https://rstudio.cloud/project/255745}{workspace}}.
\item
  Added material to the
  \textbf{\href{https://paceafenet.github.io/final_report/}{Final
  Report}}.
\end{itemize}

\hypertarget{agenda-items-2}{%
\subsection{Agenda Items}\label{agenda-items-2}}

\begin{enumerate}
\def\labelenumi{\arabic{enumi}.}
\tightlist
\item
  Discuss form for new equipment
\item
  User guide
\item
  RStudio Cloud
\item
  Final Report
\end{enumerate}

\hypertarget{march-12th---march-17th-2019}{%
\section{March 12th - March 17th,
2019}\label{march-12th---march-17th-2019}}

\hypertarget{progress-items-3}{%
\subsection{Progress Items}\label{progress-items-3}}

\begin{itemize}
\tightlist
\item
  Created a
  \textbf{\href{https://travis-shinin-spot.shinyapps.io/new_equip_form/}{form}}
  to add data for new equipment to the eTool database. Added link to it
  in eTool.
\item
  Created a project
  \textbf{\href{https://rstudio.cloud/project/255745}{workspace}} that
  could be used to build capacity in R programming so that the eTool
  could be adapted in country.
\end{itemize}

\hypertarget{agenda-items-3}{%
\subsection{Agenda Items}\label{agenda-items-3}}

\begin{enumerate}
\def\labelenumi{\arabic{enumi}.}
\tightlist
\item
  New form created - go over
\item
  Discuss conversation with Theresa and RStudio Cloud test work space.
\item
  Discuss RStudio Connect updates below.
\item
  Payment delay. Need status update, can provide further info if
  necessary.
\end{enumerate}

\hypertarget{march-4th---march-11th-2019}{%
\section{March 4th - March 11th,
2019}\label{march-4th---march-11th-2019}}

\hypertarget{progress-items-4}{%
\subsection{Progress Items}\label{progress-items-4}}

\begin{itemize}
\tightlist
\item
  Discussed current progress with Theresa. Specifically, we discussed
  issues regarding deploying the tool to the field and issues with
  partners in-country being able to fully adapt, maintain, and use data
  entered/altered using the eTool.
\item
  Installed RStudio Connect QuickStart. This is essentially a trial
  version of RStudio Connect that you run locally on your machine rather
  than on a server. This is the service I would recommend to CDC as
  other analytic tools are essentially for reporting only. You can
  develop reports and administrate access to those reports, however
  users/viewers cannot alter underlying data using those reporting
  tools.
\item
  Updated Final report.
\item
  Updated eTool to include second page with form for uploading data for
  a single new piece of equipment.
\end{itemize}

\hypertarget{agenda-items-4}{%
\subsection{Agenda Items}\label{agenda-items-4}}

\begin{enumerate}
\def\labelenumi{\arabic{enumi}.}
\tightlist
\item
  Discuss RStudio Connect layout and admin tools.
\item
  Discuss RStudio Cloud.
\item
  Discuss additions to eTool.
\end{enumerate}

\hypertarget{feb-25th---march-3rd-2019}{%
\section{Feb 25th - March 3rd, 2019}\label{feb-25th---march-3rd-2019}}

\hypertarget{progress-items-5}{%
\subsection{Progress Items}\label{progress-items-5}}

\begin{itemize}
\tightlist
\item
  I have begun developing more technical documentation on adapting the
  eTool code for future use.
\end{itemize}

\hypertarget{agenda-items-5}{%
\subsection{Agenda Items}\label{agenda-items-5}}

\begin{enumerate}
\def\labelenumi{\arabic{enumi}.}
\tightlist
\item
  Discuss eTool outline.
\item
  Discuss technical guidance.
\item
  Payment request for January submitted to CDC. Any to get more timely
  payment?
\end{enumerate}

\hypertarget{feb-11---feb-17-2019-feb-18---feb-24th}{%
\section{Feb 11 - Feb 17, 2019 + Feb 18 - Feb
24th}\label{feb-11---feb-17-2019-feb-18---feb-24th}}

\hypertarget{progress-items-6}{%
\subsection{Progress Items}\label{progress-items-6}}

\begin{itemize}
\tightlist
\item
  Developed outline for eTool (Deliverable 2). Refined the outline
  further. This outline can serve as a user manual for those tracking
  equipment data in the field. A technical user manual will complement
  this as a component of deliverable 3. I would like to share this
  document with the larger equipment maintenance group.

  \begin{itemize}
  \tightlist
  \item
    Output as pdf and added to project home page.
  \end{itemize}
\item
  Discuss whether or not I am ready to develop a technical user manual
  for maintaining the eTool code.
\end{itemize}

\hypertarget{agenda-items-6}{%
\subsection{Agenda Items}\label{agenda-items-6}}

\begin{enumerate}
\def\labelenumi{\arabic{enumi}.}
\tightlist
\item
  Discuss eTool outline.
\item
  Discuss considering Deliverable 1 as completed.
\end{enumerate}

\hypertarget{feb-4---feb-10-2019}{%
\section{Feb 4 - Feb 10, 2019}\label{feb-4---feb-10-2019}}

\hypertarget{progress-items-7}{%
\subsection{Progress Items}\label{progress-items-7}}

\begin{itemize}
\tightlist
\item
  Made eTool dev repository private and sent Mimi receipt for 12 month
  subscription. Sent email alluding to who I would like to add as
  collaborators. They will need to create GitHub accounts to view,
  discussed in email to Ruth and Fatima.
\item
  eTool dev repository set as private, however, I left the final report
  and progress reports repositories as public. There is nothing
  sensitive and the websites developed from these repositories are worth
  showing outsiders for accountability and to describe our work.
\end{itemize}

\hypertarget{agenda-items-7}{%
\subsection{Agenda Items}\label{agenda-items-7}}

\begin{enumerate}
\def\labelenumi{\arabic{enumi}.}
\tightlist
\item
  I went through the equipment inventory from Ruth that she received
  from Biologic Solutions. It could be useful if these are in the actual
  Cameroon data, however, is this all the information they have about
  their equipment? Currently, there is nothing other than type, do they
  have service dates? Or perhaps this is just info they don't have? I
  would really like to know how specific their data is.
\item
  Discuss my email regarding adding those involved with the equipment
  maintenance project as collaborators on GitHub if not resolved
  already.
\item
  \textbf{Good time to discuss SOW:} Deliverable 1 in SOW -
  \emph{Support the review of existing Nigeria/Cameroon country data
  (Deliverable -- data reviewed, interim report} - Due Date: December,
  2018

  \begin{itemize}
  \tightlist
  \item
    I have reviewed and
    \textbf{\href{https://github.com/paceafenet/etool_dev/blob/master/existing_lab_data_exploration.md}{reported}}
    on the current state of the Nigeria data from the past laboratory
    assessment.
  \item
    I also have reached out to the Cameroon Team repeatedly, and have
    not had any success in obtaining their data. I understand that
    ASLM/IRESSEF/CDC will be meeting in Nigeria soon (perhaps even this
    week) where it is possible progress could be made on this front in
    person.
  \item
    I would like to consider this deliverable fulfilled. Although, I
    would be happy to review their data if it is made accessible to me
    in the very near future. If we cannot obtain it during February I
    would like to close this deliverable and consider it completed.
  \end{itemize}
\item
  Deliverable 2 in SOW - \emph{Collaborate with and lead the IT Task
  Force to identify the best systems, processes and tools to meet the
  program goals (Deliverable -- weekly calls with Task Force to support
  review of data and development of eTool; final list of eTool features
  outlined} - Due Date: February, 2019

  \begin{itemize}
  \tightlist
  \item
    We have consistently held weekly meetings and I have provided
    updates on progress. Ruth has been the most consistent attendee on
    the progress management side and I believe my progress has been
    swift and satisfactory.
  \item
    There are a few more changes I plan to make to the eTool, however, I
    believe that the core functionality and layout is set. If it is
    agreed upon that the current tool is acceptable, I would like to
    begin creating an outline of eTool features and post this report to
    GitHub to satisfy this deliverable.
  \item
    Finalizing the eTool layout and working on the outline will allow me
    to begin developing metrics for a dashboard to support project and
    laboratory supervisors in assessing maintenance activities (part of
    SOW deliverable 3).
  \end{itemize}
\end{enumerate}

\hypertarget{jan-28---feb-3-2019}{%
\section{Jan 28 - Feb 3, 2019}\label{jan-28---feb-3-2019}}

\hypertarget{progress-items-8}{%
\subsection{Progress Items}\label{progress-items-8}}

\begin{itemize}
\tightlist
\item
  eTool Updates - Added confirmation message once the user selects the
  update date button. This will inform them that they have altered the
  underlying equipment data. Additionally, I have added popup boxes to
  inform them they have altered the underlying data.
\item
  Updated the final report, specifically the R libraries section.
\item
  Reviewed contract deliverables. Worth discussing the current state
  (see agenda items).
\end{itemize}

\hypertarget{agenda-items-8}{%
\subsection{Agenda Items}\label{agenda-items-8}}

\begin{enumerate}
\def\labelenumi{\arabic{enumi}.}
\tightlist
\item
  I received my first wire which includes the fees for subscriptions for
  GitHub and shinyapps.io. Other than those who have been included in
  the IT calls as well as the the larger equipment maintenance group, is
  there anyone else I should include on the access lists for these
  accounts?
\item
  Discuss current layout of the tool. Really coming along nicely.
\item
  Deliverable 1 in SOW - \emph{Support the review of existing
  Nigeria/Cameroon country data (Deliverable -- data reviewed, interim
  report} - Due Date: December, 2018

  \begin{itemize}
  \tightlist
  \item
    I have reviewed and
    \textbf{\href{https://github.com/paceafenet/etool_dev/blob/master/existing_lab_data_exploration.md}{reported}}
    on the current state of the Nigeria data from the past laboratory
    assessment.
  \item
    I also have reached out to the Cameroon Team repeatedly, and have
    not had any success in obtaining their data. I understand that
    ASLM/IRESSEF/CDC will be meeting in Nigeria soon (perhaps even this
    week) where it is possible progress could be made on this front in
    person.
  \item
    I would like to consider this deliverable fulfilled. Although, I
    would be happy to review their data if it is made accessible to me
    in the very near future. If we cannot obtain it during February I
    would like to close this deliverable and consider it completed.
  \end{itemize}
\item
  Deliverable 2 in SOW - \emph{Collaborate with and lead the IT Task
  Force to identify the best systems, processes and tools to meet the
  program goals (Deliverable -- weekly calls with Task Force to support
  review of data and development of eTool; final list of eTool features
  outlined} - Due Date: February, 2019

  \begin{itemize}
  \tightlist
  \item
    We have consistently held weekly meetings and I have provided
    updates on progress. Ruth has been the most consistent attendee on
    the progress management side and I believe my progress has been
    swift and satisfactory.
  \item
    There are a few more changes I plan to make to the eTool, however, I
    believe that the core functionality and layout is set. If it is
    agreed upon that the current tool is acceptable, I would like to
    begin creating an outline of eTool features and post this report to
    GitHub to satisfy this deliverable.
  \item
    Finalizing the eTool layout and working on the outline will allow me
    to begin developing metrics for a dashboard to support project and
    laboratory supervisors in assessing maintenance activities (part of
    SOW deliverable 3).
  \end{itemize}
\end{enumerate}

\hypertarget{jan-21---jan-27-2019}{%
\section{Jan 21 - Jan 27, 2019}\label{jan-21---jan-27-2019}}

\hypertarget{progress-items-9}{%
\subsection{Progress Items}\label{progress-items-9}}

\begin{itemize}
\tightlist
\item
  I was able to get the data behind the eTool (googlesheets) to update
  based on user input in the eTool. At first I had trouble reconciling
  the date formats of the dates from my original ``fake'' data with the
  new data added by the user. I was able to get things working correctly
  but my initial approach would have slowed the tool down too much even
  for demonstration purposes. I am re-worked my approach and now it is
  functioning correctly. The tool now includes core functionality either
  outlined in the contract or discussed during calls, namely it can:

  \begin{itemize}
  \tightlist
  \item
    Give a summary of the current state of the equipment
  \item
    Allow the user to update equipment data - I'll have to play around
    with the eTool behavior but I think the user has to refresh the app
    for changes to be reflected.
  \item
    The eTool will only show the most recent data for each piece of
    equipment, however the original data is retained in the underlying
    googlesheet, allowing for the development of equipment ``life
    books.''
  \end{itemize}
\item
  As a follow-up to an inquiry from Theresa, I created an example code
  that could send alerts via email alerting partners to examine the
  eTool -
  \textbf{\href{https://github.com/paceafenet/etool_dev/blob/master/email_proof_of_concept.Rmd}{See
  example}}. R library used for this (blastula) can be found in the
  \textbf{\href{https://paceafenet.github.io/final_report/}{final
  report}}. Additionally, I created a pricing guide for row level
  security as well as report scheduling features in commercial software
  - priced out RStudio Connect, PBI, Tableau. Summary and slides sent to
  Theresa (Fatima and Ruth CC'd) -
  \textbf{\href{https://github.com/paceafenet/etool_dev/blob/master/Quick\%20Pricing\%20Guide\%20\%E2\%80\%93\%20analytics\%20reporting\%20tools.pptx}{See
  Slides}}. Did she get it? Might have gone to spam.
\end{itemize}

\hypertarget{agenda-items-9}{%
\subsection{Agenda Items}\label{agenda-items-9}}

\begin{enumerate}
\def\labelenumi{\arabic{enumi}.}
\tightlist
\item
  Follow-up on any action items from larger group meeting. Discuss the
  PowerPoint in the eTool repository (see link above).
\item
  Discuss the current state of the layout and the addition of `reactive'
  events.
\item
  Briefly discuss ONA thoughts.
\item
  Consider follow-up with Cameroon team for their data. Worth getting to
  add to report of current data state. Gives us good information for
  what we `need' for the tool as well as completing one of the project
  deliverables.
\end{enumerate}

\hypertarget{jan-7---jan-13-jan-14---jan-20-2019}{%
\section{Jan 7, - Jan 13 + Jan 14, - Jan 20,
2019}\label{jan-7---jan-13-jan-14---jan-20-2019}}

\hypertarget{progress-items-10}{%
\subsection{Progress Items}\label{progress-items-10}}

\textbf{Note:} Combined last week's progress with this week's since no
one was on for the Wed IT call.

\begin{itemize}
\tightlist
\item
  As a follow-up to an inquiry from Theresa, I created an example code
  that could send alerts via email alerting partners to examine the
  eTool. R library used for this (blastula) can be found in the
  ((\href{https://paceafenet.github.io/final_report/}{final report}**.
  Additionally, I created a pricing guide for row level security as well
  as report scheduling features in commercial software - priced out
  RStudio Connect, PBI, Tableau. Summary and slides sent to Theresa
  (Fatima and Ruth CC'd).
\item
  Sent Invoice 2 for payment to Mimi. First invoice still in progress
  for payment.
\item
  I changed the UI in app. First I added a couple ways to alter the
  underlying data and also condensed the interface layout. Still in
  progress to make sure it's working properly. Retains historical data
  as well.
\item
  Finalized initial layout of the tool including background colors, text
  color, and object. Functionality to edit data to come later. Added
  custom layout for mobile. Currently, I have hidden several charts that
  appear on the desktop app from the mobile app. Will plan to change the
  UI for mobile.
\item
  Posted most up to date app to shinyapps.io.
\item
  Sent Ruth the
  \textbf{\href{https://travis-shinin-spot.shinyapps.io/etool_dev/}{link}}
  to the app.
\item
  Added eTool link to the project home page.
\item
  Presented the current layout to the larger group for feedback.
\item
  Read up on ONA document. Usable API and seems reasonable, the question
  would be WHO would take ownership in the field and be responsible for
  entering data?
\end{itemize}

\hypertarget{agenda-items-10}{%
\subsection{Agenda Items}\label{agenda-items-10}}

\begin{enumerate}
\def\labelenumi{\arabic{enumi}.}
\tightlist
\item
  Have not heard back from Mimi as of yet. Need feedback on invoice.
\item
  Follow-up on any action items from larger group meeting. Discuss the
  PowerPoint in the eTool repository (see link above).
\item
  Discuss the current state of the layout and the addition of `reactive'
  events.
\item
  Briefly discuss ONA thoughts.
\item
  Consider follow-up with Cameroon team for their data. Worth getting to
  add to report of current data state. Gives us good information for
  what we `need' for the tool as well as completing one of the project
  deliverables.
\end{enumerate}

\hypertarget{dec-31-2018---jan-6-2019}{%
\section{Dec 31, 2018 - Jan 6, 2019}\label{dec-31-2018---jan-6-2019}}

\hypertarget{progress-items-11}{%
\subsection{Progress Items}\label{progress-items-11}}

\begin{itemize}
\tightlist
\item
  Working on layout of eTool. Currently have a map, charts describing
  the current state of the data, as well as tables in a single layout.
\item
  Updated final report with additional libraries used in eTool
  development (DT and leaflet).
\end{itemize}

\hypertarget{agenda-items-11}{%
\subsection{Agenda Items}\label{agenda-items-11}}

\begin{enumerate}
\def\labelenumi{\arabic{enumi}.}
\tightlist
\item
  Current state of invoice. Sent, 12/26, need feedback on format and
  payment to progress.
\item
  Discuss eTool current state. I have made significant progress, on the
  layout and functionality of the tool. I still need to add the ability
  to edit the data and have these changes captured in the app. However,
  the layout is coming along nicely. If you are not accessing the
  meeting via a computer, here is a
  \textbf{\href{https://github.com/paceafenet/etool_dev/blob/master/screen\%20grabs/eTool\%20pic\%20one\%201_7_19.jpg}{link}}
  to a screen grab of the current layout.
\item
  See updates in the
  \textbf{\href{https://paceafenet.github.io/final_report/}{Final
  Report}} for additional tools used during development.
\item
  Have reached out to Cameroon a few times, hoping to reconnect now that
  most people should be back from leave.
\end{enumerate}

\hypertarget{dec-17---dec-30-2018}{%
\section{Dec 17 - Dec 30, 2018}\label{dec-17---dec-30-2018}}

\hypertarget{progress-items-12}{%
\subsection{Progress Items}\label{progress-items-12}}

\begin{itemize}
\tightlist
\item
  During the Dec 18 call it was requested that in addition to the
  website version of the progress reports, I also make the report
  available as a pdf or word document if possible. It took a bit of
  figuring out, but I will now be asking Ruth to attach the most up to
  date version of the progress report as a pdf to each week's meeting
  invite as well as include the link to the progress report site.
\item
  Drafted and sent the first invoice for first five weeks of work.
\item
  Finalized the Nigeria report. During the Dec 18 call it was also
  requested that I also summarize the service providers in the Nigerian
  assessment data. Note to Travis: Go to the
  \textbf{\href{https://github.com/paceafenet/etool_dev/blob/master/existing_lab_data_exploration.md}{report}}
  and summarize this additional table.
\item
  Began developing eTool application code based on fake data set.
\end{itemize}

\hypertarget{agenda-items-12}{%
\subsection{Agenda Items}\label{agenda-items-12}}

\begin{enumerate}
\def\labelenumi{\arabic{enumi}.}
\tightlist
\item
  Discuss eTool layout. So far working on the map to show what labs need
  attention (discuss criteria), general layout scheme developed.
\item
  Discuss Nigeria report updates.
\end{enumerate}

\hypertarget{dec-3---dec-16-2018}{%
\section{Dec 3 - Dec 16, 2018}\label{dec-3---dec-16-2018}}

\hypertarget{progress-items-13}{%
\subsection{Progress Items}\label{progress-items-13}}

\begin{itemize}
\tightlist
\item
  Reached out to Cameroon for in-country data
\item
  Reached out to Mimi for clarification regarding invoicing procedures.
  Will await her response before upgrading current GitHub account to
  Developer version.
\item
  My plan is to use Google Sheets as the data source for the eTool, and
  use the eTool to edit the data in Google Sheets. I will use an
  approach similar to this
  \textbf{\href{https://jennybc.shinyapps.io/10_read-write-private-sheet/}{app}}
  (code
  \textbf{\href{https://github.com/jennybc/googlesheets/tree/master/inst/shiny-examples/10_read-write-private-sheet}{here})}.
  I have authenticated the project Google account using this approach
  and can access a toy data set I created in Google Sheets.
\item
  Added
  \textbf{\href{https://github.com/paceafenet/etool_dev/blob/master/existing_lab_data_exploration.md}{Nigeria
  Data Report}} to the project home page. A link to this report is
  included here and on the home page. A more aesthetically pleasing copy
  of the report has also been sent as an attachment ahead of this
  meeting. Full process and initial results contained in this report,
  major conclusions and questions resulting from these analyses included
  below in the agenda items.
\end{itemize}

\hypertarget{agenda-items-13}{%
\subsection{Agenda Items}\label{agenda-items-13}}

\begin{enumerate}
\def\labelenumi{\arabic{enumi}.}
\tightlist
\item
  Follow-up after initial exploration of Nigeria data.
\end{enumerate}

\begin{itemize}
\tightlist
\item
  For more accurate location of lab equipment we would will need
  latitude and longitudes of labs, or at least more descriptive
  addresses. Geo coded existing addresses to limited success.
\item
  These data can be used for demonstration as they are now, but the
  eTool will be most useful with a complete inventory of equipment at
  each lab.
\item
  Information initially identified as essential including calibration
  dates which could be used to identify when equipment is due for
  maintenance, calibration, or retirement is largely dependent on
  manufacturer specifications. From the data as it is currently, it
  would not be possible to identify when equipment should next be
  serviced. This point will require further discussion.
\end{itemize}

\hypertarget{nov-22---dec-2-2018}{%
\section{Nov 22 - Dec 2, 2018}\label{nov-22---dec-2-2018}}

\hypertarget{progress-items-14}{%
\subsection{Progress Items}\label{progress-items-14}}

\begin{itemize}
\tightlist
\item
  Created a project GitHub account and associated repositories.
  Currently there are three repositories; one that will contain all
  eTool development code, another that will host a
  \textbf{\href{https://pages.github.com/}{GitHub Page}} for progress,
  and a third repository that will host a GitHub page for the project's
  \textbf{\href{https://paceafenet.github.io/final_report/}{Final
  Report}}.
\item
  Created associated RProject file for each repository and made
  templates that will be used to populate the GitHub pages referenced
  above.
\item
  Began writing the outline and portions of the Final Report page. This
  will be a living document that serves as a road map of how this
  project was completed.
\item
  Began reviewing Nigeria data.
\item
  Created `fake' equipment data should we be unable to obtain sufficient
  equipment data from the field.
\end{itemize}

\hypertarget{agenda-items-14}{%
\subsection{Agenda Items}\label{agenda-items-14}}

\begin{enumerate}
\def\labelenumi{\arabic{enumi}.}
\tightlist
\item
  It is important that I store all code and progress reports using
  GitHub both for accountability and to facilitate reproduction of this
  project in the future. It is worth considering what should be kept in
  the repository. For example, the data I receive from Nigeria and
  Cameroon in country partners may include information we would not want
  to be accessible in GitHub.
\end{enumerate}

\begin{itemize}
\tightlist
\item
  Comments on the GitHub Pages layout?
\item
  Thoughts on where to store in country data?
\end{itemize}

\begin{enumerate}
\def\labelenumi{\arabic{enumi}.}
\setcounter{enumi}{1}
\tightlist
\item
  Lab-level Metrics for evaluation;
\end{enumerate}

\begin{itemize}
\tightlist
\item
  Equipment up time vs downtime
\item
  Avg number of equipment past due for repair or servicing
\item
  Elapsed time after past due
\item
  What are acceptable levels for the above?
\end{itemize}

\begin{enumerate}
\def\labelenumi{\arabic{enumi}.}
\setcounter{enumi}{2}
\tightlist
\item
  Nigeria Data: Quite a bit there, maybe not everything we would need.
  Biggest issue; for each machine type, one row per lab, allows them to
  say more than one machine but only one serial number showing.Took a
  while to successfully import the data from the Nigeria Access
  Databases, solved it. Still working on initial data exploration, will
  update on next call.
\item
  When can we hope to obtain in-country data from Cameroon?
\item
  When should I purchase and invoice a shinyapps.io account?
\end{enumerate}

\hypertarget{sep-16---oct-31-2019}{%
\section{Sep 16 - Oct 31, 2019}\label{sep-16---oct-31-2019}}

\hypertarget{progress-items-15}{%
\subsection{Progress Items}\label{progress-items-15}}

\begin{itemize}
\tightlist
\item
  Research

  \begin{itemize}
  \tightlist
  \item
    Read through online documents for the best approach to collecting
    data using ODK technologies. My sense is that the best way to go
    about it is to use a combination of the
    \href{https://ona.io/about-us.html}{Ona} platform and
    \href{https://docs.opendatakit.org/collect-intro/}{ODK Collect} for
    Android devices.
  \item
    ODK recently released
    \href{https://docs.opendatakit.org/central-intro/}{ODK Central}, a
    way for organizations to host their data collection on a private
    server. ODK Aggregate is a similar service and can be hosted either
    on-premises or even on inexpensive cloud-based platforms like
    Digital Ocean. My impression is that Ona is very user friendly. It
    is very easy to export data from Ona in a number of formats, upload
    forms configured in Excel files as surveys, and to manage forms and
    projects through a free account.
  \item
    Researched ODK Briefcase for offline capabilities. I do not think
    ODK Briefcase is necessary for this project. Briefcase is
    essentially a local copy of a survey project directory complete with
    all forms, meta data, and responses. The advantage of using
    briefcase is that you can take responses without connection, upload
    them via SD card to a Briefcase directory on a laptop, and then
    upload the whole directory to an online project directory hosted on
    a server at a later time once you have an internet connection. I do
    not think this is necessary. ODK Collect and Ona are sufficient for
    this project. ODK Collect allows the user to capture responses in
    blank form(s). Their responses are cashed in the device browser, and
    then submitted to the Ona host once they have online connection.
  \end{itemize}
\item
  Development

  \begin{itemize}
  \tightlist
  \item
    Created forms using
    \href{https://docs.opendatakit.org/build-intro/}{ODK Build}, a tool
    created by ODK to make creating ODK forms with a Graphical User
    Interface possible. ODK forms are uploaded to platforms that accept
    forms as Excel files but must be formatted in a very specific way.
    ODK Build allows the user to easily design the appearance of a form,
    export it to Excel, and then upload that file to platforms like Ona.
    The four forms created include the Equipment Activity Form -
    documentation of Maintenance Activities, Equipment Information Form
    - to document newly added equipment and any changes to equipment
    information, Request Maintenance - to request maintenance for a
    specific piece of equipment, Request Calibration - to request
    calibration for a specific piece of equipment.
  \item
    Created a project account in Ona to manage forms and submitted dummy
    responses to these forms.
  \item
    Figured out how to pull data from Ona forms.
  \end{itemize}
\end{itemize}


\end{document}
